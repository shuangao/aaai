As we use GD to solve SDA problem, we will introduce the related work on both areas.

In SDA, many works have been proposed to utilize the unlabeled data.  \cite{yao2015semi} proposed a framework named Semi-supervised Domain Adaptation with Subspace Learning (SDASL) to correct data distribution mismatch and leverage unlabeled data. \cite{Donahue_2013_CVPR} proposed a framework for adapting classifiers by "borrowing" the source data to the target domain using a combination of available labeled and unlabeled examples. \cite{daume2010frustratingly} proposed a method by augmenting the feature space to compensate the domain shift. \cite{duan2012visual} proposed a method using the unlabeled data to measure the mismatch between the domains based on the maximum mean discrepancy.

There are also many works related to GD for computer vision tasks. \cite{Sharmanska_2013_ICCV} proposed a Rank Transfer method that uses attributes, annotator
rationales, object bounding boxes, and textual descriptions as the privileged information for object recognition. \cite{Motiian_2016_CVPR} proposed {the information bottleneck method with privileged information (IBPI)} that leverage the auxiliary information such as supplemental visual features, bounding box annotations and 3D skeleton tracking data to improve visual recognition performance. \cite{Tzeng_2015_ICCV} proposed a CNN architecture for domain adaptation to leverage the knowledge from limited or no labeled data using the soft label. \cite{urban2016deep} use a small shallow net to mimick the output of a large deep net while using layer-wised distillation with $\ell_2$ loss of the outputs of student and teacher net. Similarly, \cite{luo2016face} use $\ell_2$ loss to train a compressed student model from the teacher model for face recognition. \cite{Gupta_2016_CVPR} use supervision transfer to distill the knowledge from a trained CNN with unlabeled data or just a few labeled data.